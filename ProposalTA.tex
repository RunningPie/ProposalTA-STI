%! TEX program = xelatex
% 
% Template Proposal Tugas Akhir
% Program Studi Sistem dan Teknologi Informasi
% Sekolah Teknik Elektro dan Informatika
% Institut Teknologi Bandung
% 
% Dibuat oleh: IGB Baskara Nugraha 
% Email: baskara@itb.ac.id 
% 
% Last updated: 20 Oktober 2025
%
% Petunjuk penggunaan:
% 1. Ada 2 file utama, yaitu ProposalTA.tex (file ini) dan daftar-pustaka.bib (file daftar pustaka).
% 2. Sunting ProposalTA.tex sesuai dengan kebutuhan Anda.
% 3. Sunting atau generate isi daftar-pustaka.bib dengan referensi yang Anda gunakan, sesuai dengan format BibLaTeX.
% 4. Simpan kedua file tersebut dalam satu folder yang sama.
% 5. Kompilasi file ProposalTA.tex menggunakan XeLaTeX dan Biber (lihat urutan cara kompilasi di bawah).
% 6. Hasil kompilasi adalah file ProposalTA.pdf yang siap dicetak.
% 
% Urutan cara kompilasi (melalui command line):
% 1. xelatex ProposalTA.tex
% 2. biber ProposalTA      
% 3. xelatex  ProposalTA.tex
% 4. xelatex  ProposalTA.tex
%
% Catatan:
% - Pastikan Anda telah menginstal paket-paket LaTeX yang diperlukan, termasuk
%   biblatex-chicago dan fontspec.
% - Gunakan editor LaTeX yang mendukung XeLaTeX, seperti TeXstudio, Overleaf, atau lainnya.
% - Jika meenggunakan Visual Studio Code sebagai editor, pastikan mengatur "latex-workshop.latex.tools" dan
%   "latex-workshop.latex.recipes" untuk mendukung XeLaTeX dan Biber dengan cara menambahkan konfigurasi berikut:
%   "latex-workshop.latex.tools": [ 
%       {
%           "name": "xelatex",
%           "command": "xelatex",
%           "args": [
%               "-synctex=1",
%               "-interaction=nonstopmode",
%               "-file-line-error",
%               "%DOC%"
%           ]
%       },
%       {
%           "name": "biber",
%           "command": "biber",
%           "args": [
%               "%DOCFILE%"
%           ]
%       }
%   ],
%   "latex-workshop.latex.recipes": [
%       {
%           "name": "xelatex -> biber -> xelatex*2",
%           "tools": [
%               "xelatex",
%               "biber",
%               "xelatex",
%               "xelatex"
%           ]
%       }
%   ]
% - Untuk referensi lebih lanjut tentang penggunaan BibLaTeX dengan gaya Chicago, silakan merujuk ke dokumentasi resmi BibLaTeX.
%   https://ctan.org/pkg/biblatex-chicago
% - Untuk referensi lebih lanjut tentang penggunaan XeLaTeX dan fontspec, silakan merujuk ke dokumentasi resmi fontspec.
%   https://ctan.org/pkg/fontspec
% - Selamat menyusun proposal tugas akhir Anda!
%
\documentclass[12pt,a4paper,oneside]{book}

% ==========================================
% BASIC PACKAGES
% ==========================================
\usepackage[utf8]{inputenc}
\usepackage{fontspec}
\setmainfont{Times New Roman}
\usepackage[a4paper, left=4cm, right=3cm, top=3cm, bottom=3cm]{geometry}
\usepackage[indonesian]{babel}
\usepackage{csquotes}
\usepackage{setspace}
\onehalfspacing
\usepackage{graphicx}
\usepackage{caption}
\usepackage{subcaption}
\usepackage{hyperref}
\usepackage{fancyhdr}
\usepackage{titlesec}
\usepackage{tocloft}
\usepackage{lipsum}
\usepackage{hyperref}
\usepackage{floatrow}

\setcounter{tocdepth}{4} % kedalaman daftar isi sampai subsubbab
\setcounter{secnumdepth}{4} % kedalaman penomoran sampai subsubbab


% ==========================================
% SITASI DAN DAFTAR PUSTAKA (MENGGUNAKAN CHICAGO MANUAL OF STYLE)
% ==========================================
\usepackage[
    backend=biber,
    authordate,
    language=english,
    autolang=other
]{biblatex-chicago}

\addbibresource{daftar-pustaka.bib}

% ==========================================
% Ubah istilah bahasa Inggris di daftar pustaka ke Bahasa Indonesia
% ==========================================
\DefineBibliographyStrings{english}{
  and          = {dan},
  andothers    = {dkk.},
  editor       = {penyunting},
  editors      = {penyunting},
  translator   = {penerjemah},
  byeditor     = {disunting oleh},
  bytranslator = {diterjemahkan oleh},
  in           = {dalam},
  edition      = {edisi},
  pages        = {hal.},
  page         = {hal.},
  volume       = {vol.},
  number       = {no.},
  urlseen      = {diakses pada},
  url          = {tautan},
}

% ==========================================
% Pastikan \cite() menampilkan (Penulis Tahun)
% ==========================================
\let\oldcite\cite
\renewcommand{\cite}{\parencite}

% ==========================================
% Atur pemisah nama penulis agar lebih natural dalam Bahasa Indonesia
% ==========================================
\renewcommand*{\finalandcomma}{} % hilangkan koma sebelum 'dan'


% ==========================================
% TAMPILAN
% ==========================================
\hypersetup{
    colorlinks=true,
    linkcolor=black,
    citecolor=black,
    urlcolor=black
}

% -- No Header dan No Footer ---
\pagestyle{plain}


% ==========================================
% AWAL DOKUMEN
% ==========================================
\begin{document}

% ==========================================
% HALAMAN JUDUL
% ==========================================
\begin{titlepage}
\begin{center}

    
    \vspace*{3cm}
    
    {\Large\bfseries\uppercase{Implementasi Integrasi Penilaian Personal dan Sosial dalam Sistem Pembelajaran Adaptif yang Sejalan dengan Strategi Nasional Kecerdasan Artifisial}}\\
     \vspace{2cm}

    {\Large \textbf{Proposal Tugas Akhir}}\\


    \vspace{1cm}
    
    
    {\large Oleh}\\[0.3cm]
    \textbf{
    {\large Dama Dhananjaya Daliman}\\
    {\large 18222047}
    }\\

    \vspace{2cm}
    
    \begin{figure}[h]
    \centering
    \includegraphics[width=0.2\textwidth]{ganesha.jpg}
    \end{figure}
    
    
     \vspace{1cm}

    \textbf{
    {\large PROGRAM STUDI SISTEM DAN TEKNOLOGI INFORMASI}\\
    {\large SEKOLAH TEKNIK ELEKTRO DAN INFORMATIKA}\\
    {\large INSTITUT TEKNOLOGI BANDUNG}\\
    {\large Desember 2025}
    }
\end{center}
\end{titlepage}



% ==========================================
% LEMBAR PENGESAHAN 
% ==========================================
\newpage
\thispagestyle{empty}
\pagenumbering{gobble}
\begin{center}
  \textbf{\large LEMBAR PENGESAHAN}\\[1cm]
  \vspace*{1.5cm}
    
  {\large\bfseries\uppercase{Implementasi Integrasi Penilaian Personal dan Sosial dalam Sistem Pembelajaran Adaptif yang Sejalan dengan Strategi Nasional Kecerdasan Artifisial}}\\
     \vspace{2cm}

  {\Large \textbf{Proposal Tugas Akhir}}\\


  \vspace{1.5cm}
    
    
  {\large Oleh}\\[0.3cm]
    \textbf{
    {\large Dama Dhananjaya Daliman}\\
    {\large 18222047}
  }\\
    
  \vspace{0.5cm}
 
  {\large Program Studi Sistem dan Teknologi Informasi}\\
  {\large Sekolah Teknik Elektro dan Informatika}\\
  {\large Institut Teknologi Bandung}\\

  \vspace{1.5cm}

  Proposal Tugas Akhir ini telah disetujui dan disahkan\\ 
  di Bandung, pada tanggal XX November 2025\\[1cm]

% ==========================================
% Versi 1 pembimbing (default)
% ==========================================
	Pembimbing  \\[3cm]
	Ir. Windy Gambetta, MBA.   \\[0.2cm]
	NIP. 196404301989031005 
% ==========================================

\end{center}

\vspace{1cm}
\noindent

% ==========================================
% Jika ada 2 pembimbing TA, uncomment dan edit 
% tabular di bawah ini. Kemudian, comment out atau hapus
% bagian versi 1 pembimbing di atas.
% ==========================================

%\begin{tabular}{p{1cm}p{7cm}p{7cm}}
%   & Pembimbing 1 & Pembimbing 2 \\[3cm]
%   & Dr. Ir. John Doe, M.T. & Dr. Mary Doe, M.Sc. \\[0.2cm]
%   &  NIP. 123456789 & NIP. 987654321
%\end{tabular}



% -- Change page number style to roman ---
\pagenumbering{roman} 


% ==========================================
% DAFTAR ISI, TABEL, GAMBAR
% ==========================================
% --- DAFTAR ISI ---
\makeatletter
\renewcommand{\tableofcontents}{%
  \clearpage
  \thispagestyle{plain}% no header
  \begin{center}
    {\large\bfseries\MakeUppercase{\contentsname}\par}
  \end{center}
  \vskip 1em
  \@starttoc{toc}%
}
\makeatother

\newpage
\renewcommand{\cfttoctitlefont}{\hfill\large\bfseries\MakeUppercase}
\renewcommand{\cftaftertoctitle}{\hfill}
\tableofcontents
\addcontentsline{toc}{chapter}{DAFTAR ISI}

% --- DAFTAR GAMBAR ---
\newpage
\renewcommand{\cftloftitlefont}{\hfill\large\bfseries\MakeUppercase}
\renewcommand{\cftafterloftitle}{\hfill}
\listoffigures
\addcontentsline{toc}{chapter}{DAFTAR GAMBAR}

% --- DAFTAR TABEL ---
\newpage
\renewcommand{\cftlottitlefont}{\hfill\large\bfseries\MakeUppercase}
\renewcommand{\cftafterlottitle}{\hfill}
\listoftables
\addcontentsline{toc}{chapter}{DAFTAR TABEL}

\mainmatter
% --- FORMAT TAMPILAN JUDUL BAB, SUBBAB, JUDUL GAMBAR DAN TABEL ---
% --- Judul Bab ---
\titleformat{\chapter}[display]
      {\centering\normalfont\large\bfseries} % Commands for the entire chapter title
      {\MakeUppercase \chaptertitlename\ \thechapter}{14pt}{\large} % Chapter number format
\renewcommand\thechapter{\Roman{chapter}}
% --- Judul Subbab dan Subsubbab ---
\titleformat{\section}
	{\normalfont\bfseries}
	{\thesection}{1em}{}
\titleformat{\subsection}
	{\normalfont\bfseries}
	{\thesubsection}{1em}{}

% --- Format judul gambar dan tabel ---
\captionsetup[figure]{labelsep=space}
\captionsetup[table]{labelsep=space}
\floatsetup[table]{capposition=top}

% --- Atur indentasi paragraf ---
\setlength{\parindent}{0pt}
% -- Change page number style to arabic ---
\pagenumbering{arabic} 
      
% ==========================================
% BAB I PENDAHULUAN
% ==========================================
\chapter{PENDAHULUAN}

% --- Latar Belakang ---
\section{Latar Belakang}
Pembelajaran yang dipersonalisasi dan adaptif sedang mendorong perubahan dalam pendidikan tinggi dari yang awalnya menggunakan pendekatan pedagogis yang berpusat pada pengajar menjadi pendekatan yang berpusat pada siswa \autocite{Ryoo2021}. Menurut \textcite{Halkiopoulos2024}, pergeseran ini terjadi karena pengembangan \textit{personalized learning} (PL), yang didefinisikan sebagai pendidikan yang disesuaikan dengan keterampilan, pengetahuan, dan preferensi individu saat belajar, suatu perkembangan yang sangat dipengaruhi oleh keberadaan \textit{artificial intelligence} (AI). Pembelajaran terpersonalisasi ini memerlukan sistem \textit{e-learning} yang dapat menyesuaikan konten dan urutan penyampaiannya dengan kemampuan pelajar, suatu tugas kustomisasi yang bisa dikerjakan sangat baik oleh AI. Salah satu aspek utama lingkungan modern terpersonalisasi seperti ini adalah hadirnya \textit{adaptive assessment} (AA), suatu sistem penilaian adaptif yang secara terus-menerus menyesuaikan proses penilaiannya sesuai dengan tingkat kesulitan konten dan berdasarkan pada kinerja, preferensi, serta pengetahuan yang diekspresikan oleh peserta didik \autocite{Halkiopoulos2024}. Para peneliti dari bidang pendidikan dan ilmu komputer saat ini telah berusaha mengoptimalkan desain lingkungan pembelajaran adaptif yang kompleks ini dengan menggunakan teknik penambangan data, \textit{machine learning}, dan \textit{deep learning} yang canggih. Teknologi-teknologi ini diperlukan karena lingkungan pembelajaran adaptif perlu mengatasi tantangan personalisasi dalam proses pembelajaran manusia di dunia nyata \autocite{Minn2022}.\\

Memahami potensi transformatif kecerdasan buatan dalam pendidikan, pemerintah Indonesia telah merumuskan Strategi Nasional Kecerdasan Artifisial (Stranas KA) untuk periode 2020--2045 melalui kelompok kerja yang dibentuk oleh Badan Pengkajian dan Penerapan Teknologi \autocite{BPPT2020}. Strategi nasional ini secara eksplisit mengidentifikasi pendidikan dan penelitian sebagai salah satu dari lima bidang prioritas utama untuk pengembangan dan implementasi AI. Dokumen strategi ini menyerukan urgensi untuk meninggalkan pendekatan pedagogis yang bersifat “\textit{one size fits all}” dan secara khusus menyoroti perlunya mengeksplorasi pengembangan sistem penilaian yang adaptif sebagai solusi.\\

Beberapa teknik sudah diterapkan dalam upaya asesmen adaptif, salah satu pendekatan yang banyak digunakan adalah \textit{knowledge tracing} (KT), suatu teknik yang menggunakan data dinamis dari interaksi siswa untuk melacak perkembangan pengetahuan seiring berjalannya waktu. Ada juga model yang lebih lawas seperti \textit{Bayesian Knowledge Tracing} (BKT) yang memodelkan penguasaan keterampilan sebagai transisi antara keadaan “telah dipelajari” dan “belum dipelajari”. Selain dua teknik tersebut, saat ini ada metode yang lebih canggih dan modern, dikenal dengan \textit{Deep Knowledge Tracing} (DKT), suatu metode yang menggunakan \textit{Recurrent Neural Network} (RNN) untuk menghasilkan pelacakan perkembangan pengetahuan seperti KT, tetapi dengan akurasi yang lebih tinggi. Meski DKT ini lebih canggih dan akurat, kompleksitasnya membuat metode ini sulit diinterpretasikan dan secara umum bekerja seperti \textit{blackbox} \autocite{Minn2022}. Alternatif yang lebih praktis dan mudah diinterpretasikan adalah sistem \textit{Elo-rating} seperti dalam permainan catur. Pendekatan ini memodelkan penilaian sebagai interaksi atau “kompetisi” antara siswa dan \textit{item} pembelajaran, dalam hal ini keduanya diberi peringkat numerik yang diperbarui setelah setiap interaksi. Metode ini efisien secara komputasi, mudah diimplementasikan, dan memberikan perkiraan yang andal bahkan dengan ukuran sampel yang kecil \autocite{Vesin2022}.\\

Teknik-teknik yang dibahas di atas semuanya unggul dalam satu hal, yaitu menciptakan pembelajaran yang dipersonalisasi untuk siswa secara individual, tetapi ada risiko bahwa personalisasi yang berlebihan dapat menyebabkan siswa yang sebenarnya bisa memperoleh manfaat dari pembelajaran dalam komunitas, justru merasa terasingkan. Hal ini kemungkinan akan menghalangi manfaat yang bisa diperoleh siswa dari aspek komunal pembelajaran, suatu dilema yang secara inheren sulit diatasi oleh model-model yang berfokus pada individu seperti DKT dan sistem \textit{Elo-rating} \autocite{Halkiopoulos2024}. Strategi Nasional KA Indonesia menekankan pentingnya ekosistem kolaboratif dalam penerapan inovasi kecerdasan artifisial, yang melibatkan pemerintah, industri, akademisi, dan komunitas, yang disebut sebagai kolaborasi “\textit{Quadruple-Helix}” \autocite{BPPT2020}. Sehingga ini menunjukkan bahwa dalam rangka mengembangkan pendidikan berbasis AI dengan asesmen adaptif, sesuai dengan Stranas KA 2020--2045, perlu adanya sistem yang dapat mengelola dan mengoptimalkan interaksi dalam pembelajaran terpersonalisasi dengan skenario yang lebih luas, tidak lagi hanya berfokus pada seorang siswa tunggal.
% \item	Kondisi atau situasi topik yang dibahas beserta permasalahannya, misalnya tentang pengelolaan informasi di puskesmas daerah pedesaan dan masalah yang dihadapi.
% \item	Berbagai solusi yang telah diterapkan atau solusi yang tersedia dan memungkinkan untuk diterapkan untuk menyelesaikan masalah tersebut.
% \end{enumerate}

% --- Rumusan Masalah ---
\section{Rumusan Masalah}
Dari uraian pada latar belakang, terlihat bahwa ada kesenjangan teknologi yang jelas, yaitu ketiadaan suatu perangkat lunak yang dapat mengelola dan mengoptimalkan interaksi dalam pembelajaran terpersonalisasi dengan skenario yang lebih luas, tidak lagi hanya berfokus pada seorang siswa tunggal. Sistem yang seperti ini diperlukan, tidak hanya untuk mewujudkan pembelajaran yang terpersonalisasi, tetapi juga untuk mengelola dan mengoptimasi pembelajaran sosial dalam dinamika berkelompok. Maka dari itu, permasalahan utama yang ingin diselesaikan adalah \textbf{“Bagaimana cara mengimplementasikan teknik asesmen adaptif dalam suatu sistem perangkat lunak berbasis kecerdasan buatan yang memerhatikan aspek pembelajaran sosial dan selaras dengan Stranas KA Indonesia 2020-2045?”}
% \begin{enumerate}
% \item	Pokok persoalan dari kondisi atau situasi yang ada saat ini. Dengan kata lain, jelaskan kelemahan atau kekurangan dari kondisi, situasi, atau solusi yang dijelaskan pada latar belakang. Ini merupakan pokok rumusan masalah.
% \item	Elaborasi lebih lanjut urgensi penyelesaian masalah tersebut (mengapa penting untuk diselesaikan dan akibat yang dapat terjadi jika tidak diselesaikan).
% \item	Usulan singkat terkait dengan solusi yang ditawarkan untuk menyelesaikan persoalan.
% Penting untuk diperhatikan bahwa persoalan yang dideskripsikan pada subbab ini akan dipertanggungjawabkan di bab Evaluasi (apakah terselesaikan atau tidak).
% \end{enumerate}

% --- Tujuan ---
\section{Tujuan}
Tujuan utama dari tugas akhir ini adalah \textbf{membangun dan mengevaluasi purwarupa sistem perangkat lunak berbasis kecerdasan buatan yang memerhatikan aspek pembelajaran sosial dan selaras dengan Stranas KA Indonesia 2020-2045}. Lebih lanjut, tujuan utama tersebut bisa diturunkan menjadi tujuan-tujuan yang lebih spesifik menjadi:
\begin{enumerate}
\item Mengintegrasikan algoritma asesmen adaptif ke dalam suatu purwarupa sistem sehingga dapat secara dinamis mengkalkulasikan penilaian performa pengguna dalam penugasan individual maupun interaksi berkelompok.
\item Mengevaluasi purwarupa sistem pada sampel populasi pengguna untuk mengevaluasi fungsionalitas dan efektivitas sistem dalam asesmen adaptif.
\end{enumerate}

% --- Batasan Masalah ---
\section{Batasan Masalah}
Untuk memastikan penelitian ini tetap fokus dan dapat diselesaikan dalam jangka waktu yang ditentukan, maka ditetapkan batasan-batasan sebagai berikut:
\begin{enumerate}
  \item Platform dan Teknologi: Purwarupa perangkat lunak yang dikembangkan akan berbasis web (\textit{web-based}). Penelitian ini tidak akan mencakup pengembangan aplikasi untuk \textit{platform} lain seperti aplikasi \textit{native} Android, iOS, atau \textit{desktop}.
  \item Algoritma Asesmen: Penelitian akan berfokus pada implementasi dan integrasi satu teknik asesmen adaptif yang komputasinya efisien dan mudah diinterpretasikan. Penelitian ini tidak akan melakukan perbandingan performa antara berbagai algoritma asesmen adaptif yang kompleks (seperti \textit{Deep Knowledge Tracing}).
  \item Konten Pembelajaran: Sistem yang dibangun merupakan sebuah kerangka kerja (\textit{framework}) yang agnostik terhadap materi pelajaran. Konten yang digunakan selama tahap evaluasi (misalnya, bank soal) akan terbatas pada satu domain pengetahuan spesifik (contoh: Dasar-Dasar Algoritma dan Pemrograman) dan tidak bersifat komprehensif.
  \item Skala Evaluasi: Evaluasi fungsionalitas dan efektivitas sistem akan dilakukan pada skala terbatas, melibatkan satu kelompok sampel populasi pengguna (misalnya, satu kelas perkuliahan). Penelitian ini tidak bertujuan untuk melakukan studi longitudinal atau generalisasi hasil ke populasi yang lebih luas.
  \item Fungsionalitas Sistem: Purwarupa akan mencakup fungsionalitas inti yang dibutuhkan untuk asesmen adaptif individual dan kolaboratif, serta visualisasi data penilaian. Fitur-fitur pendukung yang biasa ditemukan pada Learning Management System (LMS) lengkap, seperti manajemen pengguna tingkat lanjut, forum diskusi, atau pengiriman notifikasi, tidak akan diimplementasikan.
\end{enumerate}
% Tuliskan batasan-batasan yang diambil dalam pelaksanaan tugas akhir. Batasan ini dapat dihindari (bersifat opsional, tidak perlu ada) jika topik atau judul tugas akhir dibuat cukup spesifik.

% --- Metodologi Pengerjaan TA ---
\section{Metodologi}

\begin{enumerate}
\item	Tahapan investigasi pengumpulan fakta di latar belakang untuk merumuskan masalah.
\item	Langkah-langkah pencarian, pengelompokan, dan penapisan literatur atau sumber informasi untuk mengumpulkan informasi yang relevan tentang topik yang diangkat, termasuk teori (konsep atau teori apa saja yang perlu dicari), hal-hal yang telah dicapai oleh orang lain (cara mencari dan kata kuncinya), dan berbagai informasi pendukung, untuk mencari solusi terhadap masalah yang dibahas. Gunakan metodologi yang tepat dalam menggali informasi dan dokumentasikan prosesnya (termasuk rekaman wawancara atau survei) di dalam Lampiran, termasuk tautan ke video atau foto. Hasil penggalian informasi ini akan dijelaskan secara sistematis di Bab II Studi Literatur.
\end{enumerate}

% ==========================================
% BAB II STUDI LITERATUR
% ==========================================
\chapter{STUDI LITERATUR}
\section{Section Pertama}
\lipsum[1]
\subsection{Subsection Pertama}
\lipsum[2]
\subsubsection{Gambar}
Penomoran subbab maksimum adalah 4 (empat) tingkat, seperti pada nomor subbab ini. Contoh gambar dapat dilihat pada Gambar \ref{gambar:jaringan}. Gambar dan judulnya diposisikan di tengah. Nomor gambar tidak diakhiri tanda titik. Gambar tersebut dibuat menggunakan aplikasi draw.io dan disimpan ke format PNG setelah dengan zoom setting pada angka 300\%. Ukuran gambar yang ditampilkan dapat diatur dengan mengubah nilai \textit{width} dalam sintaks \textit{includegraphics}.

\begin{figure}[t] % pilihan opsi yang disarankan: t = top, b = bottom, h = here
	\centering
    	\includegraphics[width=0.7\textwidth]{gambar1.png}
	\caption{Contoh gambar jaringan}
	\label{gambar:jaringan}
\end{figure}


\subsubsection{Tabel}
Contoh tabel dapat dilihat pada Tabel \ref{tbl:harga1} dan \ref{tbl:harga2}. Tabel dan judulnya dibuat rata kiri dan judul tabel diletakkan di atas tabel. Usahakan tabel dapat ditulis dalam satu halaman, tidak terpotong ke halaman berikutnya.

\begin{table}[t] % pilihan opsi yang disarankan: t = top, b = bottom, h = here
\centering
	\begin{tabular}{ | p{2cm} | p{2cm} | p{3cm} |}
	\hline
	Nama 	& Satuan 		& Harga \\
	\hline
	Buku 	& Exemplar	& 25000 \\
	Komputer	& Unit		& 2500000 \\
	Pensil	& Buah		& 118900 \\
	\hline
	\end{tabular}
\caption{Tabel harga bahan pokok}
\label{tbl:harga1}
\end{table}

\begin{table}[h] % pilihan opsi yang disarankan: t = top, b = bottom, h = here
\centering
	\begin{tabular}{ | l | c | r | }
	\hline
	Nama 	& Satuan 		& Harga \\
	\hline
	Buku 	& Exemplar	& 25000 \\
	Komputer	& Unit		& 2500000 \\
	Pensil	& Buah		& 118900 \\
	\hline
	\end{tabular}
\caption{Tabel harga bahan sekunder}
\label{tbl:harga2}
\end{table}

\subsubsection{Rumus}
Contoh rumus matematika dapat ditulis seperti pada Persamaan \ref{eq:contoh1} di bawah ini. 
Penomoran persamaan diletakkan di sebelah kanan, dan rumus ditulis dalam mode \textit{display math}.
\begin{equation}
E = mc^2
\label{eq:contoh1}
\end{equation}

\subsection{Subsection Kedua}
\lipsum[3]

% ============================================================================================
% BAB III ANALISIS MASALAH
% Pembagian subbab tidak rigid dan dapat bervariasi. Bab ini minimal berisi analisis kebutuhan
% fungsional dan nonfungsional, analisis berbagai alternatif solusi yang dapat ditawarkan, dan
% metode pemilihan solusi yang diusulkan.
% ============================================================================================
\chapter{ANALISIS MASALAH}
\section{Analisis Kondisi Saat Ini}
Menurut \textcite{BPPT2020}, gambarkan terlebih dahulu model konseptual sistem yang ada saat ini. Model konseptual ini berisi berbagai komponen atau subsitem dan interaksi antarsubsistem tersebut. Setelah itu, berikan penjelasan tentang masalah yang ada pada sistem tersebut. Paragraf berikut berisi contoh penjabaran masalah sistem informasi fasilitas kesehatan untuk pasien \autocite{Halkiopoulos2024}. 
\section{Analisis Kebutuhan}
\lipsum[4]
\subsection{Identifikasi Masalah Pengguna}
\lipsum[5]
\subsection{Kebutuhan Fungsional}
\lipsum[6]
\subsection{Kebutuhan Nonfungsional}
\lipsum[7]

\section{Analisis Pemilihan Solusi}
\subsection{Alternatif Solusi}
\lipsum[8]
\subsection{Analisis Penentuan Solusi}
\lipsum[9]

% ==========================================
% BAB IV DESAIN KONSEP SOLUSI
% ==========================================
\chapter{DESAIN KONSEP SOLUSI}
Ilustrasikan desain konsep solusi dalam bentuk model konseptual dan penjelasan secara ringkas, 
beserta perbedaannya dengan sistem saat ini. Ilustrasi harus dapat dibandingkan (\textit{before} and \textit{after}). 
Karena masih berupa proposal, bab ini hanya berisi gambar desain konsep solusi tersebut dan 
penjelasan perbandingannya dengan gambar sistem yang ada saat ini (yang tergambar di awal Bab III).

% ==========================================
% BAB V RENCANA SELANJUTNYA
% ==========================================
\chapter{RENCANA SELANJUTNYA}
Jelaskan secara detail langkah-langkah rencana selanjutnya, hal-hal yang diperlukan atau akan disiapkan, dan risiko dan mitigasinya, yang meliputi:
\begin{enumerate}
\item	Rencana implementasi, termasuk alat dan bahan yang diperlukan, lingkungan, konfigurasi, biaya, dan sebagainya.
\item	Desain pengujian dan evaluasi, misalnya metode verifikasi dan validasi.
\item	Analisis risiko dan mitigasi, misalnya tindakan selanjutnya jika ada yang tidak berjalan sesuai rencana.
\end{enumerate}


\backmatter

% ==========================================
% DAFTAR PUSTAKA
% ==========================================
\printbibliography[title={DAFTAR PUSTAKA}]

% ==========================================
% LAMPIRAN (optional)
% ==========================================
\appendix

\chapter{LAMPIRAN A: SOURCE CODE}

\chapter{LAMPIRAN B: HASIL SURVEI}


\end{document}
 